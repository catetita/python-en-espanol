% This file was generated by Nimrod.
% Generated: 2019-06-21 01:27:50 UTC
\documentclass[a4paper]{article}
\usepackage[left=2cm,right=3cm,top=3cm,bottom=3cm]{geometry}
\usepackage[utf8]{inputenc}
\usepackage[T1]{fontenc}
\usepackage{graphicx}
\usepackage{lmodern}
\usepackage{fancyvrb, courier}
\usepackage{tabularx}
\usepackage{hyperref}

\begin{document}
\title{README }
\author{}

\tolerance 1414 
\hbadness 1414 
\emergencystretch 1.5em 
\hfuzz 0.3pt 
\widowpenalty=10000 
\vfuzz \hfuzz 
\raggedbottom 

\maketitle

\newenvironment{rstpre}{\VerbatimEnvironment\begingroup\begin{Verbatim}[fontsize=\footnotesize , commandchars=\\\{\}]}{\end{Verbatim}\endgroup}

% to pack tabularx into a new environment, special syntax is needed :-(
\newenvironment{rsttab}[1]{\tabularx{\linewidth}{#1}}{\endtabularx}

\newcommand{\rstsub}[1]{\raisebox{-0.5ex}{\scriptsize{#1}}}
\newcommand{\rstsup}[1]{\raisebox{0.5ex}{\scriptsize{#1}}}

\newcommand{\rsthA}[1]{\section{#1}}
\newcommand{\rsthB}[1]{\subsection{#1}}
\newcommand{\rsthC}[1]{\subsubsection{#1}}
\newcommand{\rsthD}[1]{\paragraph{#1}}
\newcommand{\rsthE}[1]{\paragraph{#1}}

\newcommand{\rstovA}[1]{\section*{#1}}
\newcommand{\rstovB}[1]{\subsection*{#1}}
\newcommand{\rstovC}[1]{\subsubsection*{#1}}
\newcommand{\rstovD}[1]{\paragraph*{#1}}
\newcommand{\rstovE}[1]{\paragraph*{#1}}

% Syntax highlighting:
\newcommand{\spanDecNumber}[1]{#1}
\newcommand{\spanBinNumber}[1]{#1}
\newcommand{\spanHexNumber}[1]{#1}
\newcommand{\spanOctNumber}[1]{#1}
\newcommand{\spanFloatNumber}[1]{#1}
\newcommand{\spanIdentifier}[1]{#1}
\newcommand{\spanKeyword}[1]{\textbf{#1}}
\newcommand{\spanStringLit}[1]{#1}
\newcommand{\spanLongStringLit}[1]{#1}
\newcommand{\spanCharLit}[1]{#1}
\newcommand{\spanEscapeSequence}[1]{#1}
\newcommand{\spanOperator}[1]{#1}
\newcommand{\spanPunctuation}[1]{#1}
\newcommand{\spanComment}[1]{\emph{#1}}
\newcommand{\spanLongComment}[1]{\emph{#1}}
\newcommand{\spanRegularExpression}[1]{#1}
\newcommand{\spanTagStart}[1]{#1}
\newcommand{\spanTagEnd}[1]{#1}
\newcommand{\spanKey}[1]{#1}
\newcommand{\spanValue}[1]{#1}
\newcommand{\spanRawData}[1]{#1}
\newcommand{\spanAssembler}[1]{#1}
\newcommand{\spanPreprocessor}[1]{#1}
\newcommand{\spanDirective}[1]{#1}
\newcommand{\spanCommand}[1]{#1}
\newcommand{\spanRule}[1]{#1}
\newcommand{\spanHyperlink}[1]{#1}
\newcommand{\spanLabel}[1]{#1}
\newcommand{\spanReference}[1]{#1}
\newcommand{\spanOther}[1]{#1}
\newcommand{\spantok}[1]{\frame{#1}}

\rsthA{\textbf{Tutorial de python}}\label{tutorial-de-python}
\includegraphics{https://github.com/catetita/pyton-en-espanol/blob/master/1\_PXHkfdYyliqb1qCrznu5TQ.jpeg}\rsthB{\textbf{Introducción}}\label{introducción}
\textbf{¿Qué es Python?}

Python es un lenguaje de programación creado por Guido van Rossum a principios de los años 90 cuyo nombre está inspirado en el grupo de cómicos ingleses “Monty Python”. Es un lenguaje similar a Perl, pero con una sintaxis muy limpia y que favorece un código legible. Se trata de un lenguaje interpretado o de script, con tipado dinámico, fuertemente tipado, multiplataforma y orientado a objetos.

\textbf{Lenguaje interpretado o de script}

Un lenguaje interpretado o de script es aquel que se ejecuta utilizando un programa intermedio llamado intérprete, en lugar de compilar el código a lenguaje máquina que pueda comprender y ejecutar directa- mente una computadora (lenguajes compilados). La ventaja de los lenguajes compilados es que su ejecución es más rápida. Sin embargo los lenguajes interpretados son más flexibles y más portables. Python tiene, no obstante, muchas de las características de los lengua- jes compilados, por lo que se podría decir que es semi interpretado. En Python, como en Java y muchos otros lenguajes, el código fuente se traduce a un pseudo código máquina intermedio llamado bytecode la primera vez que se ejecuta, generando archivos .pyc o .pyo (bytecode optimizado), que son los que se ejecutarán en sucesivas ocasiones.

\textbf{Tipado dinámico}

La característica de tipado dinámico se refiere a que no es necesario declarar el tipo de dato que va a contener una determinada variable, sino que su tipo se determinará en tiempo de ejecución según el tipo del valor al que se asigne, y el tipo de esta variable puede cambiar si se le asigna un valor de otro tipo.

\textbf{Fuertemente tipado}

No se permite tratar a una variable como si fuera de un tipo distinto al que tiene, es necesario convertir de forma explícita dicha variable al nuevo tipo previamente. Por ejemplo, si tenemos una variable que contiene un texto (variable de tipo cadena o string) no podremos tra- tarla como un número (sumar la cadena “9” y el número 8). En otros lenguajes el tipo de la variable cambiaría para adaptarse al comporta- miento esperado, aunque esto es más propenso a errores.

\textbf{Multiplataforma}

El intérprete de Python está disponible en multitud de plataformas (UNIX, Solaris, Linux, DOS, Windows, OS/2, Mac OS, etc.) por lo que si no utilizamos librerías específicas de cada plataforma nuestro programa podrá correr en todos estos sistemas sin grandes cambios.

\textbf{Orientado a objetos}

La orientación a objetos es un paradigma de programación en el que los conceptos del mundo real relevantes para nuestro problema se tras- ladan a clases y objetos en nuestro programa. La ejecución del progra- ma consiste en una serie de interacciones entre los objetos. Python también permite la programación imperativa, programación funcional y programación orientada a aspectos.

\textbf{¿Por qué Python?}

Python es un lenguaje que todo el mundo debería conocer. Su sintaxis simple, clara y sencilla; el tipado dinámico, el gestor de memoria, la gran cantidad de librerías disponibles y la potencia del lenguaje, entre otros, hacen que desarrollar una aplicación en Python sea sencillo, muy rápido y, lo que es más importante, divertido. La sintaxis de Python es tan sencilla y cercana al lenguaje natural que los programas elaborados en Python parecen pseudocódigo. Por este motivo se trata además de uno de los mejores lenguajes para comenzar a programar. Python no es adecuado sin embargo para la programación de bajo nivel o para aplicaciones en las que el rendimiento sea crítico. Algunos casos de éxito en el uso de Python son Google, Yahoo, la NASA, Industrias Light \& Magic, y todas las distribuciones Linux, en las que Python cada vez representa un tanto por ciento mayor de los programas disponibles.

\textbf{Instalación de Python}

Existen varias implementaciones distintas de Python: CPython, Jython, IronPython, PyPy, etc. CPython es la más utilizada, la más rápida y la más madura. Cuando la gente habla de Python normalmente se refiere a esta implementación. En este caso tanto el intérprete como los módulos están escritos en C. Jython es la implementación en Java de Python, mientras que IronPython es su contrapartida en C\# (.NET). Su interés estriba en que utilizando estas implementaciones se pueden utilizar todas las librerías disponibles para los programadores de Java y .NET. PyPy, por último, como habréis adivinado por el nombre, se trata de una implementación en Python de Python. CPython está instalado por defecto en la mayor parte de las distribu- ciones Linux y en las últimas versiones de Mac OS. Para comprobar si está instalado abre una terminal y escribe python. Si está instalado se iniciará la consola interactiva de Python y obtendremos parecido a lo siguiente:

\begin{rstpre}
\spanIdentifier{Python} \spanFloatNumber{2.5}\spanOperator{.}\spanDecNumber{1} \spanPunctuation{(}\spanIdentifier{r251}\spanPunctuation{:}\spanDecNumber{54863}\spanPunctuation{,} \spanIdentifier{May} \spanDecNumber{2} \spanDecNumber{2007}\spanPunctuation{,} \spanDecNumber{16}\spanPunctuation{:}\spanDecNumber{56}\spanPunctuation{:}\spanDecNumber{35}\spanPunctuation{)}
\spanPunctuation{\symbol{91}}\spanIdentifier{GCC} \spanFloatNumber{4.1}\spanOperator{.}\spanDecNumber{2} \spanPunctuation{(}\spanIdentifier{Ubuntu} \spanFloatNumber{4.1}\spanOperator{.}\spanDecNumber{2}\spanOperator{-}\spanDecNumber{0}\spanIdentifier{ubuntu4}\spanPunctuation{)}\spanPunctuation{\symbol{93}} \spanIdentifier{on} \spanIdentifier{linux2}
\spanKeyword{Type} \spanIdentifier{“help”}\spanPunctuation{,} \spanIdentifier{“copyright”}\spanPunctuation{,} \spanIdentifier{“credits”} \spanKeyword{or} \spanIdentifier{“license”} \spanKeyword{for} \spanIdentifier{more}
\spanIdentifier{information}\spanOperator{.}
\spanOperator{>>>}
\end{rstpre}
La primera línea nos indica la versión de Python que tenemos ins- talada. Al final podemos ver el prompt (>>>) que nos indica que el intérprete está esperando código del usuario. Podemos salir escribiendo exit(), o pulsando Control + D. Si no te muestra algo parecido no te preocupes, instalar Python es muy sencillo. Puedes descargar la versión correspondiente a tu sistema ope- rativo desde la web de Python, en \href{http://www.python.org/download/}{http://www.python.org/download/}. Existen instaladores para Windows y Mac OS. Si utilizas Linux es muy probable que puedas instalarlo usando la herramienta de gestión de paquetes de tu distribución, aunque también podemos descargar la aplicación compilada desde la web de Python.

\rsthB{\textbf{Herramientas básicas}}\label{herramientas-básicas}
Existen dos formas de ejecutar código Python. Podemos escribir líneas de código en el intérprete y obtener una respuesta del intérprete para cada línea (sesión interactiva) o bien podemos escribir el código de un programa en un archivo de texto y ejecutarlo. A la hora de realizar una sesión interactiva os aconsejo instalar y uti- lizar iPython, en lugar de la consola interactiva de Python. Se puede encontrar en \href{http://ipython.scipy.org/}{http://ipython.scipy.org/}. iPython cuenta con características añadidas muy interesantes, como el autocompletado o el operador ?. (para activar la característica de autocompletado en Windows es nece- sario instalar PyReadline, que puede descargarse desde \href{http://ipython}{http://ipython}. scipy.org/ moin/PyReadline/Intro) La función de autocompletado se lanza pulsando el tabulador. Si escribimos fi y pulsamos Tab nos mostrará una lista de los objetos que comienzan con fi (file, filter y finally). Si escribimos file. y pulsamos Tab nos mostrará una lista de los métodos y propiedades del objeto file. El operador ? nos muestra información sobre los objetos. Se utiliza añadiendo el símbolo de interrogación al final del nombre del objeto del cual queremos más información. Por ejemplo:

\begin{rstpre}
\spanKeyword{In} \spanPunctuation{\symbol{91}}\spanDecNumber{3}\spanPunctuation{\symbol{93}}\spanPunctuation{:} \spanIdentifier{str}\spanOperator{?}
\spanKeyword{Type}\spanPunctuation{:} \spanKeyword{type}
\spanIdentifier{Base} \spanIdentifier{Class}\spanPunctuation{:}
\spanIdentifier{String} \spanIdentifier{Form}\spanPunctuation{:}
\spanIdentifier{Namespace}\spanPunctuation{:} \spanIdentifier{Python} \spanIdentifier{builtin}
\spanIdentifier{Docstring}\spanPunctuation{:}
\spanIdentifier{str}\spanPunctuation{(}\spanKeyword{object}\spanPunctuation{)} \spanOperator{->} \spanIdentifier{string}

\spanKeyword{Return} \spanIdentifier{a} \spanIdentifier{nice} \spanIdentifier{string} \spanIdentifier{representation} \spanKeyword{of} \spanIdentifier{the} \spanKeyword{object}\spanOperator{.}
\spanKeyword{If} \spanIdentifier{the} \spanIdentifier{argument} \spanKeyword{is} \spanIdentifier{a} \spanIdentifier{string}\spanPunctuation{,} \spanIdentifier{the} \spanKeyword{return} \spanIdentifier{value} \spanKeyword{is} \spanIdentifier{the} \spanIdentifier{same}
\spanKeyword{object}\spanOperator{.}
\end{rstpre}
En el campo de IDEs y editores de código gratuitos PyDEV \emph{(http:// pydev.sourceforge.net/)} se alza como cabeza de serie. PyDEV es un plu- gin para Eclipse que permite utilizar este IDE multiplataforma para programar en Python. Cuenta con autocompletado de código (con información sobre cada elemento), resaltado de sintaxis, un depurador gráfico, resaltado de errores, explorador de clases, formateo del código, refactorización, etc. Sin duda es la opción más completa, sobre todo si instalamos las extensiones comerciales, aunque necesita de una canti- dad importante de memoria y no es del todo estable.

Otras opciones gratuitas a considerar son SPE o Stani’s Python Editor \emph{(http://sourceforge.net/projects/spe/)} ,  Eric \emph{(http://die-offenbachs.de/eric/)} , BOA Constructor \emph{(http://boa-constructor.sourceforge.net/)} o incluso emacs o vim.

Si no te importa desembolsar algo de dinero, Komodo \emph{(http://www. activestate.com/komodo\_ide/)} y Wing IDE \emph{(http://www.wingware.com/)} son también muy buenas opciones, con montones de características interesantes, como PyDEV, pero mucho más estables y robustos. Ade- más, si desarrollas software libre no comercial puedes contactar con Wing Ware y obtener, con un poco de suerte, una licencia gratuita para Wing IDE Professional :)

\rsthA{\textbf{Mi primer programa en python}}\label{mi-primer-programa-en-python}
Como comentábamos en el capítulo anterior existen dos formas de ejecutar código Python, bien en una sesión interactiva (línea a línea) con el intérprete, o bien de la forma habitual, escribiendo el código en un archivo de código fuente y ejecutándolo. El primer programa que vamos a escribir en Python es el clásico Hola Mundo, y en este lenguaje es tan simple como:

\begin{rstpre}
\spanIdentifier{print} \spanIdentifier{“Hola} \spanIdentifier{Mundo”}
\end{rstpre}
Vamos a probarlo primero en el intérprete. Ejecuta python o ipython según tus preferencias, escribe la línea anterior y pulsa Enter. El intér- prete responderá mostrando en la consola el texto Hola Mundo. Vamos ahora a crear un archivo de texto con el código anterior, de forma que pudiéramos distribuir nuestro pequeño gran programa entre nuestros amigos. Abre tu editor de texto preferido o bien el IDE que hayas elegido y copia la línea anterior. Guárdalo como hola.py, por ejemplo. Ejecutar este programa es tan sencillo como indicarle el nombre del archivo a ejecutar al intérprete de Python

\begin{rstpre}
\spanIdentifier{python} \spanIdentifier{hola}\spanOperator{.}\spanIdentifier{py}
\end{rstpre}
pero vamos a ver cómo simplificarlo aún más.

Si utilizas Windows los archivos .py ya estarán asociados al intérprete de Python, por lo que basta hacer doble clic sobre el archivo para eje- cutar el programa. Sin embargo como este programa no hace más que imprimir un texto en la consola, la ejecución es demasiado rápida para poder verlo si quiera. Para remediarlo, vamos a añadir una nueva línea que espere la entrada de datos por parte del usuario.

\begin{rstpre}
\spanIdentifier{print} \spanIdentifier{“Hola} \spanIdentifier{Mundo”}
\spanIdentifier{raw\_input}\spanPunctuation{(}\spanPunctuation{)}
\end{rstpre}
De esta forma se mostrará una consola con el texto Hola Mundo hasta que pulsemos Enter.

Si utilizas Linux (u otro Unix) para conseguir este comportamiento, es decir, para que el sistema operativo abra el archivo .py con el intérprete adecuado, es necesario añadir una nueva línea al principio del archivo:

\begin{rstpre}
\spanComment{\#!/usr/bin/python}
\spanIdentifier{print} \spanIdentifier{“Hola} \spanIdentifier{Mundo”}
\spanIdentifier{raw\_input}\spanPunctuation{(}\spanPunctuation{)}
\end{rstpre}
A esta línea se le conoce en el mundo Unix como \texttt{shebang} , \texttt{hashbang} o \texttt{sharpbang} . El par de caracteres \#! indica al sistema operativo que dicho script se debe ejecutar utilizando el intérprete especificado a continuación. De esto se desprende, evidentemente, que si esta no es la ruta en la que está instalado nuestro intérprete de Python, es necesario cambiarla.

Otra opción es utilizar el programa env (de environment, entorno) para preguntar al sistema por la ruta al intérprete de Python, de forma que nuestros usuarios no tengan ningún problema si se diera el caso de que el programa no estuviera instalado en dicha ruta:

\begin{rstpre}
\spanComment{\#!/usr/bin/env python}
\spanIdentifier{print} \spanIdentifier{“Hola} \spanIdentifier{Mundo”}
\spanIdentifier{raw\_input}\spanPunctuation{(}\spanPunctuation{)}
\end{rstpre}
Por supuesto además de añadir el shebang, tendremos que dar permi- sos de ejecución al programa.

\begin{rstpre}
\spanIdentifier{chmod} \spanOperator{+}\spanIdentifier{x} \spanIdentifier{hola}\spanOperator{.}\spanIdentifier{py}
\end{rstpre}
Y listo, si hacemos doble clic el programa se ejecutará, mostrando una consola con el texto Hola Mundo, como en el caso de Windows.

También podríamos correr el programa desde la consola como si trata- ra de un ejecutable cualquiera:

\begin{rstpre}
\spanOperator{./}\spanIdentifier{hola}\spanOperator{.}\spanIdentifier{py}
\end{rstpre}
En Python los tipos básicos se dividen en:

\begin{itemize}\item Números, como pueden ser 3 (entero), 15.57 (de coma flotante) o
\end{itemize}
7 + 5j (complejos)

\begin{itemize}\item Cadenas de texto, como “Hola Mundo”
\item Valores booleanos: True (cierto) y False (falso).
\end{itemize}
Vamos a crear un par de variables a modo de ejemplo. Una de tipo cadena y una de tipo entero:

\begin{rstpre}
\spanComment{\# esto es una cadena}
\spanIdentifier{c} \spanOperator{=} \spanIdentifier{“Hola} \spanIdentifier{Mundo”}
\spanComment{\# y esto es un entero}
\spanIdentifier{e} \spanOperator{=} \spanDecNumber{23}
\spanComment{\# podemos comprobarlo con la función type}
\spanKeyword{type}\spanPunctuation{(}\spanIdentifier{c}\spanPunctuation{)}
\spanKeyword{type}\spanPunctuation{(}\spanIdentifier{e}\spanPunctuation{)}
\end{rstpre}
Como veis en Python, a diferencia de muchos otros lenguajes, no se declara el tipo de la variable al crearla. En Java, por ejemplo, escribiría- mos:

\begin{rstpre}
\spanIdentifier{String} \spanIdentifier{c} \spanOperator{=} \spanIdentifier{“Hola} \spanIdentifier{Mundo”}\spanPunctuation{;}
\spanIdentifier{int} \spanIdentifier{e} \spanOperator{=} \spanDecNumber{23}\spanPunctuation{;}
\end{rstpre}
Este pequeño ejemplo también nos ha servido para presentar los comentarios inline en Python: cadenas de texto que comienzan con el carácter \# y que Python ignora totalmente. Hay más tipos de comenta- rios, de los que hablaremos más adelante.

\rsthB{\textbf{Números}}\label{números}
Como decíamos, en Python se pueden representar números enteros, reales y complejos.

\textbf{Enteros}

Los números enteros son aquellos números positivos o negativos que no tienen decimales (además del cero). En Python se pueden repre- sentar mediante el tipo int (de integer, entero) o el tipo long (largo). La única diferencia es que el tipo long permite almacenar números más grandes. Es aconsejable no utilizar el tipo long a menos que sea necesario, para no malgastar memoria.

El tipo int de Python se implementa a bajo nivel mediante un tipo long de C. Y dado que Python utiliza C por debajo, como C, y a dife- rencia de Java, el rango de los valores que puede representar depende de la plataforma.

En la mayor parte de las máquinas el long de C se almacena utilizando 32 bits, es decir, mediante el uso de una variable de tipo int de Python podemos almacenar números de -231 a 231 - 1, o lo que es lo mismo, de

\begin{description}
\item[-2.147.483.648 a 2.147.483.647. En plataformas de 64 bits, el rango es] 
\end{description}
de -9.223.372.036.854.775.808 hasta 9.223.372.036.854.775.807.

El tipo long de Python permite almacenar números de cualquier preci- sión, estando limitados solo por la memoria disponible en la máquina.

Al asignar un número a una variable esta pasará a tener tipo int, a menos que el número sea tan grande como para requerir el uso del tipo long.

\begin{rstpre}
\spanComment{\# type(entero) devolvería int}
\spanIdentifier{entero} \spanOperator{=} \spanDecNumber{23}
\end{rstpre}
También podemos indicar a Python que un número se almacene usan- do long añadiendo una L al final:

\begin{rstpre}
\spanComment{\# type(entero) devolvería long}
\spanIdentifier{entero} \spanOperator{=} \spanDecNumber{23}\spanIdentifier{L}
\end{rstpre}
El literal que se asigna a la variable también se puede expresar como un octal, anteponiendo un cero:

\begin{rstpre}
\spanComment{\# 027 octal = 23 en base 10}
\spanIdentifier{entero} \spanOperator{=} \spanDecNumber{027}
\end{rstpre}
o bien en hexadecimal, anteponiendo un 0x:

\begin{rstpre}
\spanComment{\# 0×17 hexadecimal = 23 en base 10}
\spanIdentifier{entero} \spanOperator{=} \spanDecNumber{0}\spanIdentifier{×17}
\end{rstpre}
\textbf{Reales}

Los números reales son los que tienen decimales. En Python se expre- san mediante el tipo float. En otros lenguajes de programación, como C, tenemos también el tipo double, similar a float pero de mayor precisión (double = doble precisión). Python, sin embargo, implementa su tipo float a bajo nivel mediante una variable de tipo double de C, es decir, utilizando 64 bits, luego en Python siempre se utiliza doble precisión, y en concreto se sigue el estándar IEEE 754: 1 bit para el signo, 11 para el exponente, y 52 para la mantisa. Esto significa que los valores que podemos representar van desde ±2,2250738585072020 x 10-308 hasta ±1,7976931348623157×10308.

La mayor parte de los lenguajes de programación siguen el mismo esquema para la representación interna. Pero como muchos sabréis esta tiene sus limitaciones, impuestas por el hardware. Por eso desde Python 2.4 contamos también con un nuevo tipo Decimal, para el caso de que se necesite representar fracciones de forma más precisa. Sin embargo este tipo está fuera del alcance de este tutorial, y sólo es necesario para el ámbito de la programación científica y otros rela- cionados. Para aplicaciones normales podeis utilizar el tipo float sin miedo, como ha venido haciéndose desde hace años, aunque teniendo en cuenta que los números en coma flotante no son precisos (ni en este ni en otros lenguajes de programación).

Para representar un número real en Python se escribe primero la parte entera, seguido de un punto y por último la parte decimal.

\begin{rstpre}
\spanIdentifier{real} \spanOperator{=} \spanFloatNumber{0.2703}
\end{rstpre}
También se puede utilizar notación científica, y añadir una e (de expo- nente) para indicar un exponente en base 10. Por ejemplo:

\begin{rstpre}
\spanIdentifier{real} \spanOperator{=} \spanFloatNumber{0.1e-3}
\end{rstpre}
sería equivalente a 0.1 x 10-3 = 0.1 x 0.001 = 0.0001

\textbf{Complejos}

Los números complejos son aquellos que tienen parte imaginaria. Si no conocías de su existencia, es más que probable que nunca lo vayas a necesitar, por lo que puedes saltarte este apartado tranquilamente. De hecho la mayor parte de lenguajes de programación carecen de este tipo, aunque sea muy utilizado por ingenieros y científicos en general.

En el caso de que necesitéis utilizar números complejos, o simplemen- te tengáis curiosidad, os diré que este tipo, llamado complex en Python, también se almacena usando coma flotante, debido a que estos núme- ros son una extensión de los números reales. En concreto se almacena en una estructura de C, compuesta por dos variables de tipo double, sirviendo una de ellas para almacenar la parte real y la otra para la parte imaginaria.

Los números complejos en Python se representan de la siguiente forma:

\begin{rstpre}
\spanIdentifier{complejo} \spanOperator{=} \spanFloatNumber{2.1} \spanOperator{+} \spanFloatNumber{7.8}\spanIdentifier{j}
\end{rstpre}
\textbf{Operadores}

Veamos ahora qué podemos hacer con nuestros números usando los operadores por defecto. Para operaciones más complejas podemos recurrir al módulo \texttt{math} .

\textbf{Operadores aritméticos}

\begin{table}\begin{rsttab}{|X|X|X|}
\hline
\textbf{Operador} & \textbf{Descripción} & \textbf{Ejemplo}\\
\hline
+ & SUMA & r = 3 + 2    \# r es 5\\
\hline
- & RESTA & r = 4 - 7    \# r es -3\\
\hline
\end{rsttab}\end{table}\begin{table}\begin{rsttab}{|X|X|X|X|X|X|X|}
\hline
\textbf{Operador} & \textbf{Descripción} & \textbf{Ejemplo}\\
\hline
- & Negación & r  = -7       \# r es -7\\
\hline
* & Multiplicación & r = 2 * 6    \# r es 12\\
\hline
** & Exponente & r = 2 ** 6   \# r es 64\\
\hline
/ & División & r = 3.5 / 2  \# r es 1.75\\
\hline
// & División entera & r = 3.5 // 2 \# r es 1.0\\
\hline
\% & Módulo & r = 7 \% 2    \# r es 1\\
\hline
\end{rsttab}\end{table}


\end{document}
